%
%	Theorieteil
%

\pagebreak
\section{Infrastructure as Code}

\onehalfspacing

\subsection{Terraform}

All major cloud providers have their infrastructure scripting tools, but there's one declarative tool that's available for all infrastructure platforms, on-premise or public, Terraform by HashiCorp.\footnote{See \textit{HashiCorp (2019)}: Deliver infrastructure as code with Terraform. \cite{terraform}}.

As of Rancher 2.3, released in October 2012, Rancher has a Terraform provider, and cluster creation and decommissioning can easily be performed from a Terraform plan as part of a move of IT to IaC.\footnote{See \textit{Rancher Labs (2019)}: Introducing the Rancher 2 Terraform Provider. \cite{terraformProvider}}

With Terraform, creating infrastructure becomes as easy as writing a piece of code. Terraform integrates nicely with your existing source code revision systems, such as GitHub or GitLab. It will also make ensure that all infrastructure deployments are repeatable and uniformly executed.

Adding Terraform to the container run-time tool chest of Rancher and Kubernetes is a major benefit and step forward, it will make creation of new Kubernetes clusters so much easier.\footnote{See \textit{Frank, C. (2020)}: Deploy Kubernetes Clusters on Microsoft Azure with Rancher. \cite{deployAzure}}

\subsection{Templates}

Another important piece for your Kubernetes tool chest are templates. Templates will allow you to predefine certain configuration items across the organization, which will help a lot when dealing with hardening in the next chapter.

Rancher offers templates on a number of levels:

\begin{itemize}
\item Cloud Credentials
\item Node Templates
\item Cluster Templates
\end{itemize}

We'll look at each of these in more detail below, but let's first have a look at the different possibilities of creating a Kubernetes cluster from within Rancher.

\subsection{Kubernetes Cluster Creation}

\subsubsection{Managed Kubernetes}

Overall there are three options to create a Kubernetes cluster with Rancher on any of the big public cloud providers:

\begin{itemize}
\item Managed Kubernetes (GKE, AKS, EKS)
\item Rancher node-driver (Azure, AWS)
\item Custom nodes (GCP, Azure, AWS)
\end{itemize}

To create a managed Kubernetes cluster, you'll have to follow the following steps:

First, within a  Terraform plan, define the Rancher provider. Then, in the Rancher cluster definition, define the GKE, AKS or EKS options and finally have Terraform create the cluster. Terraform will be using the platform API for this.

In this setup, the Kubernetes control plane will be managed by the cloud provider and Rancher's functionality will be somewhat limited.

\subsubsection{Rancher Node-Driver}

If you're happy to have Rancher manage the control plane and have full control, and are on either AWS or Azure, consider using the Rancher node-driver. To do this, following the following steps:

First, within a  Terraform plan, define the Rancher provider. Second, in the Rancher cluster definition, define Azure or AWS cloud provider within the RKE cluster options. Then, create the appropriate cloud credentials and node templates. Finally, have Terraform create the cluster, using docker-machine.

This is the most preferred option, and the one that we will follow throughout the document.

\subsubsection{Custom Nodes}

If you need more fine-grained control over the underlying infrastructure, Rancher also offers the ability to use custom created nodes for its Kubernetes clusters. For this, you'll need to follow these steps:

First, within a  Terraform plan, define both the Rancher provider and an infrastructure provider. Second, in the Rancher cluster definition, define the cloud provider within the RKE cluster options. Then, have Terraform create the cluster nodes with the infrastructure provider and pass the Rancher registration command to cloud-init

Although this is the most flexible approach, Rancher will loose the ability to horizontally scale the node pools, among other features. It also requires that the Terraform plans have access to credentials for the infrastructure provider, which could provide in Rancher centrally in the option above.

\subsubsection{Import}

A fourth, and final option, would be to create clusters completely outside of Terraform and Rancher and import them later. This is a quite valid option, but outside of the scope of our document as it would require other forms of automation.

\subsection{Rancher Provider}

The first step is to set up the credentials for Rancher. All access to Rancher is controlled from the built-in RBAC controller and every use can create their own API keys.

"Picture"

We'll then take the defined token and create the Rancher provider in our plan file \verb|provider.tf|:

\begin{lstlisting}[caption=Rancher Provider, frame=single, basicstyle=\ttfamily]
# Rancher
provider "rancher2" {
  api_url = var.rancher-url
  token_key = var.rancher-token
}
\end{lstlisting}

This is all the end-user credential and provider setup we'll need to create the Kubernetes clusters. On \verb|terraform init| the Rancher provider will be downloaded and initialized. 

\subsection{Cloud Credentials}

It is good practice to keep the provider definitions separate from the main plan, from here on all example plan code will go into the main plan file, \verb|main.tf|. Also, from here on, the example plan code will be for a deployment on Microsoft Azure, but can be easily adapted for Amazon Web Services, for example.

The first template we want to define are the cloud credentials. Regardless of the cloud provider, we need to proved access credentials. These credentials could be created by the user themselves, or better, be preprovisioned by the IT organization.

In our Terraform plan, creating the credentials looks like this:

\begin{lstlisting}[caption=Cloud Credentials, frame=single, basicstyle=\ttfamily]
# Rancher cloud credentials
resource "rancher2_cloud_credential" "credential_az" {
  name = "Azure Credentials"
  azure_credential_config {
    client_id = var.az-client-id
    client_secret = var.az-client-secret
    subscription_id = var.az-subscription-id
  }
}
\end{lstlisting}

To create credentials we can also use the Rancher GUI with the same input fields:

\begin{figure}[H]
\centering
\caption {Cloud Credentials}
\includegraphics[width=\linewidth]{images/cloud-credentials.png}
\label{fig:cloudCredentials}
\end{figure}

\subsection{Node Templates}

A Kubernetes cluster consists of one or more node pools, at least one for the control plane and one or more for the worker nodes. On a small installation, the control plane and the workers can reside on the same nodes.

To create a node pool, we need to define node templates in our plan:

\begin{lstlisting}[caption=Node Template, frame=single, basicstyle=\ttfamily]
# Rancher node template
resource "rancher2_node_template" "template_az" {
  name = "Azure Node Template"
  cloud_credential_id = rancher2_cloud_credential...id
  engine_install_url = var.dockerurl
  azure_config {
    disk_size = var.disksize
    image = var.image
    location = var.az-region
    managed_disks = true
    open_port = var.az-portlist
    resource_group = var.az-resource-group
    storage_type = var.az-storage-type
    size = var.type
  }
}
\end{lstlisting}

In this template, we define size, OS image and regional placement of the node. Depending on the future role, we can define them as small or as big as we need them, or define nodes with specialized hardware, such as GPUs, for machine-learning tasks.

We can make the same definition of node pools as above with the Rancher GUI:

\begin{figure}[H]
\centering
\caption {Node Template}
\includegraphics[width=\linewidth]{images/node-template.png}
\label{fig:nodeTemplate}
\end{figure}

\subsection{Cluster Templates}

The third and final template that we are going to define is the template for the actual cluster.

\begin{lstlisting}[caption=Cluster Template, frame=single, basicstyle=\ttfamily]
# Rancher cluster template 
resource "rancher2_cluster_template" "template_az" {
  name = "Azure Cluster Template"
  template_revisions {
    name = "v1"
    default = true
    cluster_config {
      cluster_auth_endpoint {
        enabled = false
      }
      rke_config {
        kubernetes_version = var.k8version
        ignore_docker_version = false
        cloud_provider {
          name = "azure"
          azure_cloud_provider {
            aad_client_id = var.az-client-id
            aad_client_secret = var.az-client-secret
            subscription_id = var.az-subscription-id
            tenant_id = var.az-tenant-id
            resource_group = var.az-resource-group
          }
        }
      }
    }
  }
}
\end{lstlisting}

In this template we define all the basic characteristics of our Kubernetes clusters that we want to enforce across the IT organization. In the example above we do not allow that users can bypass Rancher and access the Kubernetes clusters directly, this will give us a solid first line of defense against malicious attacks based on access credentials.

If you prefer the Rancher GUI, you can create cluster templates there, too:

\begin{figure}[H]
\centering
\caption {Cluster Template}
\includegraphics[width=\linewidth]{images/cluster-template.png}
\label{fig:clusterTemplate}
\end{figure}

\subsection{Kubernetes Cluster}

Now we have all the components together to build a Kubernetes cluster, credentials, node and cluster templates. 

A fresh Kubernetes cluster is only a few lines in our plan:

\begin{lstlisting}[caption=Kubernetes Cluster, frame=single, basicstyle=\ttfamily]
# Rancher cluster
resource "rancher2_cluster" "cluster_az" {
  name         = "az-${random_id.instance_id.hex}"
  description  = "Terraform"
  cluster_template_id = rancher2_cluster_template.template_az.id
  cluster_template_revision_id = rancher2_cluster_template.template_az.default_revision_id
  depends_on = [rancher2_cluster_template.template_az]
}
\end{lstlisting}

We give our cluster a name and a description, and link it to the three templates we defined before. That's all that it takes!

In the Rancher GUI, creating a cluster is equally easy:

"Picture"

With predefined templates, creating well-defined Kubernetes clusters is no longer difficult at all and does not require any in-depth knowledge of Kubernetes.

After successful creation, Rancher provides a dashboard for the newly created cluster:

\begin{figure}[H]
\centering
\caption {Rancher Dashboard}
\includegraphics[width=\linewidth]{images/cluster-dashboard-new.png}
\label{fig:clusterOverview}
\end{figure}

This figure above shows the new Rancher dashboard layout, as introduced in March 2020 with Rancher 2.4.
